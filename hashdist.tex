\documentclass[sans,mathserif]{beamer}
%handout,notes=show

%\usepackage{pgfpages}
%\pgfpagesuselayout{8 on 1}[a4paper,border shrink=5mm]

\usetheme{default}
\usepackage{fp}
\usepackage[thicklines]{cancel}
\usepackage{tikz}
\usepackage{multirow}
\usepackage{amsmath}
\usepackage{ifthen}
\usepackage{animate}
\usepackage{setspace}
\usepackage{forloop}
\usepackage{tikz}

%\usepackage{concmath}
%\usepackage{pxfonts}
%\usepackage{eulervm}
%\usepackage{mathpazo}
%\renewcommand\mathfamilydefault{\rmdefault}

%\usefonttheme{professionalfonts}
%\setmathfont{}
%\setsansfont{Palatino}
\usetikzlibrary{arrows,backgrounds,positioning,fit,chains,shapes,calc}

%\usepackage{handoutWithNotes}
%\pgfpagesuselayout{4 on 1 with notes}[a4paper,border shrink=5mm]

\setbeamertemplate{navigation symbols}{}

\title{Hashdist -- Yet Another Desperate Attempt at Fixing Scientific Software Distribution}
\author{Dag Sverre Seljebotn\\Ond\v{r}ej \v{C}ert\'{i}k\\Chris Kees}
%\institute{Department of Mathematics \\ University of Oslo}
\date{Simula, January 24, 2013}

\newcommand{\V}{\vskip1em}

\setbeamersize{sidebar width left=0cm, sidebar width right=0cm}
\setbeamersize{text margin left=.8cm, text margin right=.8cm}

\defbeamertemplate{note page}{infolines}
{%
  \vskip3em
%  \setstretch{1.8}
  \Large
  \rmfamily
  \insertnote
}
\setbeamertemplate{note page}[infolines]

\renewcommand{\CancelColor}{\color{red}}

\begin{document}


\begin{frame}
  \titlepage

  \begin{center} {\tt http://github.com/hashdist}
  \end{center}
~

{\footnotesize

Dag Sverre's work for Simula Innovation is funded by a grant from the \\
International Research Office, US Army Engineer Research and Development Center
(BAA contract W911NF-12-1-0604)
}

\end{frame}

\begin{frame}
  \begin{center}
    {\LARGE The Problem}
  \end{center}
\end{frame}


\begin{frame}[fragile]
  \frametitle{HPC software distribution}
\uncover<+->{What makes HPC so special?}
  \begin{itemize}
  \item<+-> Only sysadmins have root (so no Debian/Ubuntu, RedHat)
    \begin{itemize}
    \item others: virtualization
    \end{itemize}
  \item<+-> Common to compile specifically for CPU for optimal speed
    \begin{itemize}
    \item others: don't care so much
    \end{itemize}
  \item<+-> Users lack computer skills (relative to what they do)
    \begin{itemize}
    \item others: either performance isn't important, or own a datacenter
    \end{itemize}
  \item<+-> Fortran compilers from multiple vendors with incompatible ABI
    \begin{itemize}
    \item others: Fortran unheard of, only one compiler vendor anyway, C ABI standard
    \end{itemize}
\item<+->
{\small
  \begin{semiverbatim}
\$ ls /usr/mpi/*
/usr/mpi/gcc:
mvapich-1.2.0  mvapich-1.2.0-qlc  mvapich2-1.7
mvapich2-1.7-qlc  openmpi-1.4.3  openmpi-1.4.3-qlc
/usr/mpi/intel:
mvapich-1.2.0-qlc  mvapich2-1.7-qlc  openmpi-1.4.3-qlc
/usr/mpi/pgi:
mvapich-1.2.0-qlc  mvapich2-1.7-qlc  openmpi-1.4.3-qlc
  \end{semiverbatim}
}
\end{itemize}
\end{frame}



% Definition of circles
\def\firstcircle{(0,0) circle (2cm)}
\def\secondcircle{(0:3cm) circle (2cm)}

\colorlet{circle edge}{blue!50}
\colorlet{circle area}{blue!20}

\tikzset{filled/.style={fill=circle area, draw=circle edge, thick},
    outline/.style={draw=circle edge, thick}}

\begin{frame}
  \frametitle{Scientific Python}

  \begin{itemize}
  \item<+-> Python applications/scripts have large dependency trees
    \begin{itemize}
    \item Really a result of {\em better} software engineering practices...
    \end{itemize}

  \item<+-> Python packaging created by web developers who mostly write Python and some C...

  \item<+-> ...while scientific Python really a shell around C, C++, Fortran codes

%  \item<+-> Others have great solutions but either HPC-style performance isn't needed (R)
%    or there's less binding to native code (Haskell, Java)

  \end{itemize}

~

\uncover<+->{

  \begin{center}
    \begin{tikzpicture}
      \begin{scope}
        \clip \firstcircle; \fill[filled] \secondcircle;
      \end{scope}
      \draw[outline] \firstcircle node {Python}; \draw[outline]
      \secondcircle node {HPC}; \node[anchor=south] at (current
      bounding box.north) {};
      % \node at (8, 2) {
      % \begin{minipage}{6cm} {\bf BONUS:} HPC sysadmins don't care
      %   about Python
      % \end{minipage}
      % };
    \end{tikzpicture}
  \end{center}

}

\end{frame}

\begin{frame}
  \frametitle{Sophisticated systems}

  \begin{itemize}
  \item<+-> Debian/Ubuntu/RedHat/Gentoo: Need root access
  \item<+-> MacPorts etc.: Mac only
  \item<+-> Gentoo Prefix: Experimental
  \item<+-> Grand Unified Builder, 0install: Lack scientific packages
  \item<+-> Python environment provided by sysadmins: Outdated
  \end{itemize}

\end{frame}

\begin{frame}
  \frametitle{Simple user-space systems}
  Sage, python-hpcmp, dorsal, EasyBuild, Hans Petter's Python script:

  \begin{itemize}
  \item<+-> Simple (good) $\Leftrightarrow$ lack features (minor inconvenience)
  \item<+-> Easy to do oneself $\Rightarrow$ difficult for one to get
    momentum
  \item<+-> The details are different for everybody
   \begin{itemize}
    \item Best LAPACK for Sage is not best LAPACK for EPD/Anaconda
    \end{itemize}
  \end{itemize}
\end{frame}

\begin{frame}
  \frametitle{Manual builds}

  \begin{itemize}
  \item<+-> {\tt ./configure -\,{}-prefix=\$HOME/local; make; make install}
  \item<+->{Or in the case of my home institute,\\
{\tt export LD\_LIBRARY\_PATH=\textasciitilde{}oldstudent/local/lib}}
  \end{itemize}

~

\uncover<+->{Some problems:}

\begin{itemize}
\item<+-> Steep learning curve (and hard-to-find bugs)
\item<+-> Tedious to move between clusters
\item<+-> Reproducibility 
\end{itemize}
\end{frame}

\begin{frame}
  \frametitle{The real knot}

Combinatorial explosion:

~

~

\begin{tikzpicture}[scale=3]
    \draw[thick,->,black] (0,0,0) -- (2,0,0) node[anchor=north east]{Compiler vendor/version};
    \draw[thick,->] (0,0,0) -- (0,1,0) node[anchor=north east]{MPI};
    \draw[thick,->] (0,0,0) -- (0,0,1) node[anchor=east]{LAPACK};
\end{tikzpicture}

~

$\times$ Python version $\times$ NumPy version $\times$ FFT library...

\end{frame}


\begin{frame}
  \frametitle{Curation}
  
  \begin{itemize}
  \item<+-> Debian, RedHat, cluster sysadmins, dorsal is all about curated software stacks
  \item<+-> Perhaps you want 60\% curated, 20\% bleeding edge or manually configured, 20\% your own code...
  \item<+-> Ubuntu + root access: Make your own PPA
  \item<+-> Build from source
  \item<+-> dorsal, EasyBuild: Contribute your own configuration
    \begin{itemize}
    \item {\tt dolfin.package} vs. {\tt dolfin-intel.package}
    \end{itemize}
  \end{itemize}
\end{frame}


\begin{frame}
  \begin{center}
    {\LARGE Hashdist}
  \end{center}
\end{frame}

% \begin{frame}
%   \frametitle{Hashdist}
% In order of priority:
%   \begin{enumerate}
%   \item<+-> ``Hash-based'' approach (based on ideas in Nix)
%   \item<+-> Deterministic/``functional'' builds for improved reproducibility
%   \item<+-> Managing combinatorial explosion
%   \end{enumerate}
% \end{frame}


\begin{frame}[fragile]
  \frametitle{Hash-based installation}

  \begin{itemize}
  \item<+-> Linux laptop:
{\small
\begin{semiverbatim}
/usr/lib/libhdf5.so
\end{semiverbatim}
}
  \item<+-> HPC environment modules:
{\small
\begin{semiverbatim}
/cluster/software/VERSIONS/hdf5-1.6.1/lib/libhdf5.so
/cluster/software/VERSIONS/hdf5-1.6.1_intel/lib/libhdf5.so
/cluster/software/VERSIONS/hdf5-1.6.1_pgi/lib/libhdf5.so
/cluster/software/VERSIONS/hdf5-1.8.9/lib/libhdf5.so
/cluster/software/VERSIONS/hdf5-1.8.9_intel/lib/libhdf5.so
/cluster/software/VERSIONS/hdf5-1.8.9_pgi/lib/libhdf5.so
\end{semiverbatim}
}
    \item<+-> Hashdist:
{\small 
\begin{semiverbatim}
~/.hdist/opt/hdf5/{\color{red}efn3}/lib/libhdf5.so
~/.hdist/opt/hdf5/{\color{red}i7ni}/lib/libhdf5.so
~/.hdist/opt/hdf5/{\color{red}qgpd}/lib/libhdf5.so
\end{semiverbatim}
}
{\footnotesize (really {\tt hdf5/efn3i7ni7lbtik4frlb5wcnqgpdmi3ql})}
  \end{itemize}
\end{frame}

\begin{frame}
  \begin{center}
    (demo)
  \end{center}
\end{frame}

\begin{frame}[fragile]
  \frametitle{Using a hash-based software store}
\uncover<+->{{\bf (1)} Describe the build ({\bf internal protocol!})}
{\footnotesize
\begin{semiverbatim}
\{ "import" : [\{ "id" : "gcc/apyicmxgafb564zz7rwhwvon7padvxdx"\},
               \{ "id" : "unix/kmbvakqpvmhl65d2unzmjupxjbjkxge2"\},
               \{ "id" : "zlib/wbg27phinbgwjg4nasb4xzf3ypo72otn"\}],
   "sources" : [\{ "key" : "tar.bz2:7jxgwn5xs5xnvsdaomvypridodr35or2"\}],
   "script" : [
        ["LDFLAGS=-L$\{ZLIB\}/lib -Wl,-rpath,$\{ZLIB\}/lib"],
        ["CFLAGS=-I$\{ZLIB\}/include\only<3->{{\color{red} -O0 -g}}"],
        ["./configure", "--prefix=$\{ARTIFACT\}", "--with-pic"]
        ["make"],
        ["make", "install"]]
\}
\end{semiverbatim}
}

~

\uncover<+->{{\bf (2)} Hash the build spec $\rightarrow$
\only<2>{{\tt \footnotesize hdf5/u4vsabroylchvmwoxf5mdpxidd4lnrwl}}%
\only<3->{{\color{red} \tt \footnotesize hdf5/fjczhadqtyx6jlbnvzlthrzsex7wz7xb}}
}

~

\uncover<4->{{\bf (3)} If not found, do an {\em isolated} build on the fly}

\end{frame}

\begin{frame}[fragile]
  \frametitle{Using a hash-based software store}
{\bf (4)} When all packages are built reate symbolic links to
  package contents in a {\em profile}:
{
\footnotesize

\begin{semiverbatim}
\$ ls -l ~/local
local -> /home/dagss/.hdist/opt/profile/w6gp

\$ ls -l /home/dagss/.hdist/opt/profile/w6gp/bin
h5copy -> /home/dagss/.hdist/opt/hdf5/whfk/bin/h5copy
h5ls -> /home/dagss/.hdist/opt/hdf5/whfk/bin/h5ls
...

\$ ldd /home/dagss/.hdist/opt/hdf5/whfk/lib/libhdf5.so
linux-vdso.so.1 => (0x00007fffeb3ff000)
libsz.so.2 => {\color{red}/home/dagss/.hdist/opt/szip/5a5t/lib/libsz.so.2}
libz.so.1 => {\color{red}/home/dagss/.hdist/opt/zlib/cll6/lib/libz.so.1}
libm.so.6 => /lib/x86\_64-linux-gnu/libm.so.6
libc.so.6 => /lib/x86\_64-linux-gnu/libc.so.6
/lib64/ld-linux-x86-64.so.2
\end{semiverbatim}
}
  
\end{frame}




\begin{frame}
  \frametitle{Consequences of hash-based installation}
  \begin{enumerate}
  \item<+-> More dimensions! Even \\
{\footnotesize
{\tt .../h5py-hdf5\_1.8.9\_pgi-python2.7-numpy1.6.3\_debug\_ubuntu12.10}
}
is not exhaustive; ``{\tt h5py/5ffg...}'' caters for everything

  \item<+-> For free: Atomic upgrades
  \item<+-> Much easier: Uninstall (GC rather than file tracking)
  \item<+-> Jump around in software history (or between branches) in seconds!
  \end{enumerate}

~

\uncover<+->{
  \begin{center}
    {\bf Sophisticated features with simple implementation}
  \end{center}
~

Prior art: Eelco Dolstra's PhD thesis/the Nix project
}
\end{frame}


\begin{frame}
  \frametitle{The tradeoff: Builds must be ``functional''}

  \begin{itemize}
  \item<+-> All dependencies and all of the environment must be
    taken into account when describing the build
  \item<+-> More hassle, but very good for reproducability
  \end{itemize}

~
 
\uncover<+->{Some help:}

~

\begin{itemize}
\item<+-> Integrate with host system (Debian, environment modules,
  ``generic'') to specify dependencies on package on host system
\item<+-> ``hdist-jail'' can issue warnings if a build process
  accesses files it shouldn't (or hide them)

\end{itemize}

\end{frame}

%\begin{frame}[fragile]
%\frametitle{What is Hashdist?}
%\begin{itemize}
%\item<+-> One of the goals is to make a reusable {\em core} for hashed-functional builds
%\item<+-> However, existing packaging systems are not focused on
%exploring combinatorial explosion
%\end{itemize}
%\end{frame}


\begin{frame}[fragile]
\frametitle{Frontend: Software stack definitions (in progress)}
Declarative approach (because you can git it and share it):

{\small

\begin{semiverbatim}
include:
  - sources # pull in ./sources.yml
  - build
  - when cluster == "abel":
    - abel-overrides
profiles:
  - name: "default"
    configuration:
      lapack_type: "openblas"
      cluster: "hexagon"
    select:
      - project: "hdf5"
        version: 1.8.2\only<2>{{\color{red} to 1.8.5 # with integer linear programming}}
      - project: "h5py"
        ...
\end{semiverbatim}
}
\end{frame}

\begin{frame}[fragile]
\frametitle{Domain-specific language focused on overrides}

Manage the combinatorial explosion without creating
packages for {\tt hdf5\_intel\_mpich}, {\tt hdf5\_gcc\_openmpi}, ...:
{

\begin{semiverbatim}
rules:
  ...
  CFLAGS: ["-g", "-O\$optlevel"]
  when recipe == "configure-make-install":
    optlevel: 2
  when project == "hdf5":
    recipe: "configure-make-install"
    when version == 1.5.2:
      optlevel: 0
    build_deps:
      - project: "zlib"
        version: 1.2.5 to 1.2.7
  ...
\end{semiverbatim}
}
\end{frame}

\begin{frame}[fragile]
  \frametitle{Build description pipeline}


Results in a set of attributes per package:
{\small
\begin{semiverbatim}
dict(
 package='hdf5',
 version='1.8.10',
 recipe='configure-make-install',
 downloads=['http://www.hdfgroup.org/ftp/HDF5/current/'
            'src/hdf5-1.8.10.tar.bz2'],
 sources=['tar.bz2:7jxgwn5xs5xnvsdaomvypridodr35or2'],
 configure=['--prefix=${ARTIFACT}', '--with-pic'],
 CFLAGS=['-O2'],
 jail='warn',
 build_deps=[zlib, unix, gcc]
)
\end{semiverbatim}
}
\end{frame}


\begin{frame}[fragile]
  \frametitle{Build description pipeline}

Feed it through a Python pipeline:
{\footnotesize

\begin{semiverbatim}
@pipeline.add\_recipe('configure-make-install')
def configure\_make\_install\_recipe(ctx, cfg, build):
    build['build']['script'].extend([
        ["LDFLAGS=$(hdist", "build-ldflags", ")"],
        ["CFLAGS=$(hdist", "build-cflags", ")"],
        ["CFLAGS+= " + " ".join(cfg.CFLAGS)],
        ['./configure'] + cfg.configure,
        ['make'],
        ['make', 'install']
        ])
\end{semiverbatim}
}

\end{frame}

\begin{frame}[fragile]
  \frametitle{Build description pipeline}

To get build specification:
{\footnotesize

\begin{semiverbatim}
\{ "import" : [\{ "id" : "gcc/apyicmxgafb564zz7rwhwvon7padvxdx"\},
               \{ "id" : "unix/kmbvakqpvmhl65d2unzmjupxjbjkxge2"\},
               \{ "id" : "zlib/wbg27phinbgwjg4nasb4xzf3ypo72otn"\}],
   "sources" : [\{ "key" : "tar.bz2:7jxgwn5xs5xnvsdaomvypridodr35or2"]\},
   "script" : [
        ["LDFLAGS=$(hdist", "build-ldflags", ")"],
        ["CFLAGS=$(hdist", "build-cflags", ")"],
        ["CFLAGS+= -O2"],
        ["./configure", "--prefix=$\{ARTIFACT\}"]
        ["make"],
        ["make", "install"]]
\}
\end{semiverbatim}
}

\end{frame}


\end{document}